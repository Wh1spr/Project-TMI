\documentclass[12pt]{article}
\usepackage[utf8]{inputenc}
\usepackage[dutch]{babel}
\usepackage{algorithm}
\usepackage{algorithmic}
\usepackage{amssymb}
\usepackage{amsmath}
\usepackage{graphicx}

\author{Wolfgang M\"ollmann \& Robbe Degr\`eve}
\title{Toepassingen van meetkunde in de informatica \\ Project: Bepaling van het Dichtste Puntenpaar}

\begin{document}

\maketitle

\newpage

\section{Beschrijving van opstelling van puntenverzameling}
Voor het opstellen van een input-file met een aantal gegeven parameters om willekeurge punten te creëren gebruiken we de functie "makeRandom".
Deze functie zal de volgende parameters als input vragen:
\begin{enumerate}
    \item inFile: De puntenverzameling zal in dit bestand weggeschreven worden, indien het bestand niet bestaat zal het aangemaakt worden.
    \item alg: het nummer van het algoritme dat gebruikt moet worden: 1 (eenvoudige algoritme), 2 (eerste variante van het doorlooplijnalgoritme) of 3 (tweede variante van het doorlooplijnalgoritme)
    \item dim: de dimensie van de punten $M \geq 2$
		\item size: het aantal punten N
\end{enumerate}

Deze functie heeft als taak om een inputbestand op te stellen met willekeurge co\"ordinaten.
De functie zal de gegeven parameters eerst neerschrijven in de inFile.
Daarna gebeurd een initialisatie van \texttt{Random r}.
Daarna zullen we per regel van het bestand itereren van 0 tot en met de dimensie en voor iedere dimensie een willekeurige waarde neerschrijven in het bestand.
De willeukeurige co\"ordinaten kunnen waardes aannemen tussen $[0, 5[$.


\begin{algorithm}
\caption{De javafunctie \texttt{makeRandom}}
\begin{algorithmic}
\STATE \textbf{Input:} $inFile$, $alg$, $dim$, $size$
\STATE infile.println(alg)
\IF {$dim < 2$}
	\PRINT Dimension needs to be greater or equal to 2.
	\STATE \texttt{Exit}
\ENDIF
\STATE Random r = new Random()
\STATE $Iterator<Double> i$ = r.doubles(size * dim, 0.0, 5.0).iterator()
\STATE $String outString = ""$
\WHILE {$i.hasNext()$}
	\STATE $outString = ""$
	\FOR {$j = 0$; $j < dim$; j++}
		\STATE outString += String.format(Locale.US, "\%17.16f ", i.next())
	\ENDFOR
	\STATE inFile.println($outString$)
\ENDWHILE
\STATE inFile.flush()
\STATE inFile.close()
\end{algorithmic}
\end{algorithm}

\newpage

\section{Analyse van de verschillende algoritmes}

\subsection{Brute-force algorithm}

\subsubsection{hoogniveau beschrijving}

\begin{algorithm}
\caption{Bereken het dichtste puntenpaar met brute-force}
\begin{algorithmic}
	\STATE \textbf{Input:}  $rij$: Array met $N$ punten (gesorteerd naar stijgende x-co\"ordinaat)
  \STATE $d$ = $+\infty$
	\STATE $dpp1$ = 0, $dpp2$ = 0
  \STATE $currentDist$ = 0
  \FOR {$i$ to $length(rij)-1$}
    \FOR {$j = i + 1$ to $length(rij)$}
      \STATE $currentDist$ = calculate\_dist($rij[i]$, $rij[j]$)
      \IF {$currentDist < d$}
				\STATE $dpp1$ = $rij[i]$
				\STATE $dpp2$ = $rij[j]$
        \STATE $d$ = $currentDist$
  		\ENDIF
    \ENDFOR
  \ENDFOR
  \RETURN $dpp1$, $dpp2$, $d$
\end{algorithmic}
\end{algorithm}

\newpage
\subsection{Variant 1 algoritme}
\subsubsection{hoogniveau beschrijving}

\begin{algorithm}
\caption{Bereken het dichtste Puntenpaar volgens variant 1}
\begin{algorithmic}
	\STATE \textbf{Input:}  $rij$: Array met $N$ punten (gesorteerd naar stijgende x-co\"ordinaat)
	\STATE $d$ = $+\infty$
	\STATE $dpp1$ = 0, $dpp2$ = 0
	\STATE $currentDist$ = 0
	\FOR {$i = 1$ to $length(rij)$}
		\FOR {$j = i - 1$ to $0$}
		\IF {$rij[i].x - rij[j].x > d$}
			\STATE $break$
		\ENDIF
		\STATE $currentDist$ = calculate\_dist($rij[i]$, $rij[j]$)
			\IF {$currentDist < d$}
				\STATE $dpp1$ = $rij[i]$
				\STATE $dpp2$ = $rij[j]$
				\STATE $d$ = $currentDist$
			\ENDIF
		\ENDFOR
	\ENDFOR
	\RETURN $dpp1$, $dpp2$, $d$
\end{algorithmic}
\end{algorithm}

\subsubsection{rekencomplexiteit}
We zullen dus nagaan dat de rekencomplexiteit van dit algoritme gelijk is aan $O(N*Kgem)$.
We kunnen in het hoogniveau algoritme zien dat de eerste for-lus zal itereren over $N-1$ punten.
Voor elk van die punten zullen we dus de afstand bepalen van de punten die binnen de verticale strook V.
Indien we het $K_{gem}$ bepalen (uitgemiddeld over $i = 1,...,N$) zullen we een goede schatting hebben hoeveel punten er worden vergeleken in iedere iteratie van de doorlooplijn.
We rekenen bij $K_{gem}$ ook het eerste element bij die een $K$ zal hebben van 0, dus kunnen we bij onze doorloplijn iteratie het eerste punt ook meetellen.
We kunnen dus concluderen dat de rekencomplexiteit van variant 1 gelijk zal zijn aan $O(N*Kgem)$.

\subsubsection{worst-case puntenverzameling}
Voor de \textit{worst-case} puntenverzameling voor de eerste variant van het doorlooplijnalgoritme voor $M = 2$ hebben we een javafunctie \texttt{makeWorstCase}.
In deze functie maken een puntenverzameling aan van één kolom, waarbij alle punten dezelfde x-waarde hebben.
De y-waarde speelt hierbij geen rol.
In onze rij-implementatie sorteren we de punten met dezelfde x-waarden volgens de y-waarde.
Het tweede punt zal dus enkel vergelijken met het punt boven hem.
Het derde punt zal nadien vergelijken met het eerste en het tweede punt.
Uiteindelijk zal het laatste (onderste) punt zal dus met alle punten vergelijken.
Dit zal ons uiteindelijk $\frac{(n+1)}{2}$ vergelijkingen geven. De rekencomplexiteit zal dus $O(n^2)$ zijn.
Voor Kavg kunnen we concluderen dat per N het aantal kavg voor de worst-case gelijk zal zijn aan $\frac{N-1}{2}$.
Kavg zal dus voor de worst-case in 2 dimesies een stijgend karakter hebben, wat in tegenstelling is tot willekeurige puntenverzamelingen in 2 dimensies (zie figuur 3).
Dat is ook logisch. We hebben daarnet geconcludeerd dat de rekentijd van variant 1 van de worst-case zal toenemen in verband met het aantal punten.
De rekencomplexiteit is gelijk aan $(N*K_{avg})$, hieruit kunnen we zien dat voor een stijgende rekencomplexiteit we ook een stijgende Kavg nodig zullen hebben.
Kmax zal ook een stijgend karakter hebben. We kunne afleiden dat voor iedere N de Kmax steeds gelijk zal zijn aan $N-1$.
Voor de worst-case zal het laatste punt dat gecontroleerd moet worden de afstand berekenen met alle vorige punten.

\newpage
\subsection{Variant 2 algorithm}
\subsubsection{hoogniveau beschrijving}

\begin{algorithm}
\caption{Bereken het dichtste Puntenpaar volgens variant 2}
\begin{algorithmic}
	\STATE \textbf{Input:}  $rij$: Array met $N$ punten (gesorteerd naar stijgende x-co\"ordinaat)
	\STATE $d$ = $+\infty$
	\STATE $dpp1$ = 0, $dpp2$ = 0
	\STATE $currentDist$ = 0
	\STATE $t$: gegevensstructuur waarin punten links van de doorlooplijn opgeslagen zijn, gesorteerd naar stijgende y-coordinaat
  \FOR {$i = 1$ to $length(rij)-1$}
    \STATE voegtoe($t, rij[i-1]$)
    \STATE $low = rij[i]$
    \STATE $low.y = low.y - d$
    \STATE $next = boven(t,low)$
    \WHILE {$next \neq Null \textrm{ AND } next.y < rij[i].y + d$}
      \IF {$next.x < rij[i].x - d$}
        \STATE verwijder($t, next$)
        \STATE $next = boven(t,next)$
        \STATE $continue$
      \ENDIF
      \STATE $currentDist$ = calculate\_dist($rij[i]$, $next$)
  	  \IF {$currentDist < d$}
  			\STATE $dpp1$ = $rij[i]$
  			\STATE $dpp2$ = $next$
  			\STATE $d$ = $currentDist$
  		\ENDIF
      \STATE $next = boven(t,next)$
    \ENDWHILE
  \ENDFOR
\end{algorithmic}
\end{algorithm}

\subsubsection{Opmerking}
Als we tijdens de behandeling van $p_i$ tegenkomen dat buiten de verticale
strook V ligt, heeft het zin om dit punt uit de zoekboom te verwijderen. Deze bewerking neemt
$O(logK)$ elementaire operaties in beslag, en kan voor iedere keer dat het punt nogmaals in de horizontale strook terecht komt $O(logK)$ elementaire operaties (voor het zoeken) besparen. Door het verwijderen van deze elementen wordt $K_{gem}$ ook veel kleiner, en dus worden er minder \textit{boven} en \textit{onder} bewerkingen uitgevoerd. \\
Het is niet nodig, maar zal het aantal bewerkingen wel verminderen.

\section{Grafieken}
\textit{Alle grafieken zijn achteraan te vinden.}

\subsection{veronderstelling}
Indien we spreken over rekencomplexiteit bedoelen we de complexiteit van het algoritme zonder de rekentijd van het sorteren van de input.
Bij tijdscomplexiteit maken we de som tussen rekentijd van het sorteren en de rekentijd van het algoritme zelf.

\subsection{figuur 1}
In Figuur 1 zullen we de rekentijden tussen het brute-force algoritme en variant 1 vergelijken.
We zullen op de x-as het aantal punten uitzetten.
De gekozen aantal punten zijn logaritmisch bepaald met als basis $1.1$ beginnend vanaf 302 tot en met 12279.
Op de y-as zetten we de rekentijd uit in milliseconden. Hiervoor gebruiken we een logaritmische functie met basis $2$.
We kunnen concluderen uit deze plot dat variant 1 voor alle onze gekozen punten een efficiënter algoritme is.
We kunnen ook zien dat de het \textit{brute-force} algoritme een steiler verloop heeft, hieruit kunnen we afleiden dat voor zeer grote aantallen punten het verschil tussen de twee algoritmes steeds groter zal worden.

\subsection{figuur 2}
In figuur 2 zullen we het verband tussen de $K_{max}$ en het aantal punten plotten voor 2 dimensies.
Op de x-as zetten we het aantal punten uit van 250 tot 12279 met een logaritmische functie met basis $1.1$.
Op de y-as zetten we de $K_{max}$ uit die in dit geval zal gaan van een minimum van 2 tot en met een maximum van 25.
We kunnen op de plot zien dat de grafiek $K_{max}$ een licht stijgend karakter heeft.
We kunnen wel zien dat steeds $K_{max}$ relatief klein zal blijven ten opzichte van het aantal punten.

\subsection{figuur 3 en 4}
In figuur 3 zullen we het verband tussen de $K_{avg}$ en het aantal punten plotten voor 2 dimensies.
Op de x-as zetten we het aantal punten uit van 250 tot 12279 met een logaritmische functie met basis $1.1$.
Op de y-as zetten we de $K_{avg}$ uit.
Op deze plot is duidelijk te herkennen dat $K_{avg}$ ongeveer constant zal blijven ongeacht het aantal punten.
In ons experiment was deze constante $1.35$.
We kunnen concluderen dat voor willekeurige puntenverzameling in 2 dimensies de $K_{avg}$ zeer klein zal zijn in vergelijkign met $N$.

In figuur 4 zullen we het verband tussen de $K_{avg}$ en het aantal punten plotten voor 3 dimensies.
Op de x-as zetten we het aantal punten uit van 250 tot 12279 met een logaritmische functie met basis $1.1$.
Op de y-as zetten we de $K_{avg}$ uit.
Op deze plot zien we een groot contrast als we vergelijken met figuur 3.
Terwijl wij bij figuur 3 spraken van een een relatief kleine Kavg voor een willekeurge aantal punten, hebben we hier een positief lineair verband.
We zullen in drie dimensies significant meer afstanden moeten berekenen dan in twee dimensies.
We kunnen dus zeggen dat vanaf 3 dimensies de efficiëntie van het algoritme al zal afnemen.

\subsection{figuur 5 en figuur 6}
In figuur 5 zullen we het verband tussen de $K_{avg}$ en het aantal dimensies plotten.
Op de x-as zetten we het aantal dimensies uit van 2 tot 19.
Op de y-as zetten we de $K_{avg}$ uit.
Op deze plot is duidelijk te herkennen dat $K_{avg}$ een positief lineair verband heeft met het aantal dimensies.
Dit zal invloed hebben op de rekentijd van variant 1 voor hogere dimensies.
In figuur 6 zullen we het aantal dimensies plotten met de rekentijd van variant 1 voor 2500 punten.
We zullen op de x-as het aantal dimensies uitzetten van 2 tot en met 19.
Op de y-as zetten we de rekentijd uit in milliseconden.
We zien in figuur 6 ook dat voor hogere dimensies het voordeel van een doorlooplijn verloren gaat.
we kunnen dit ook afleiden uit figuur 5 en de rekencomplexiteit van variant 1, namelijk $O(N*K_{avg})$.
Omdat $K_{avg}$ een positief lineair verband heeft met het aantal dimensies zal ook de rekentijd in hogere dimensies een positiefe trend vertonen.

\newpage
In figuur 3 en 4 kunnen we al zien dat bij dimensie 2 $K_{avg}$ ongeveer constant blijft, terwijl bij dimensie 3 er een positief verband is tussen $K_{avg}$ en het aantal punten.
Hierbij gaat een deel van het voordeel verloren, maar het is nog steeds beter dan het \textit{brute-force} algoritme, omdat we nog steeds uit 1 dimensie 'selecteren' voor de verticale strook.
Het voordeel zal er dus altijd zijn, t.o.v. \textit{brute-force}, maar zal veel kleiner worden voor stijgende M.

\subsection{figuur 7}
In figuur 7 zullen we het verband tussen het aantal dimensies en $K_{max}$ plotten.
Op de x-as zetten we het aantal dimensies uit van 2 tot 19.
Op de y-as zetten we de $K_{max}$ uit.
We kunnen een duidelijk positief lineair verband tussen $K_{max}$ en het aantal dimensies waarnemen.

\newpage

\begin{figure}
\includegraphics[width=\textwidth]{Simpel-var1-rekentijd.png}
\caption{De plot van de rekentijden tussen het brute-force algoritme en variant 1 voor 2 dimensies}
\end{figure}

\begin{figure}
\includegraphics[width=\textwidth]{punten-Kmax.png}
\caption{De plot tussen het aantal punten en Kmax voor 2 dimensies}
\end{figure}

\begin{figure}
\includegraphics[width=\textwidth]{punten-Kavg}
\caption{De plot tussen het aantal punten en Kavg voor 2 dimensies}
\end{figure}

\begin{figure}
\includegraphics[width=\textwidth]{punten-Kavgdim3.png}
\caption{De plot tussen het aantal punten en Kavg voor 3 dimensies}
\end{figure}

\begin{figure}
\includegraphics[width=\textwidth]{dim-var1-rekentijd.png}
\caption{De plot tussen het aantal dimensies en de rekentijd van variant 1 (voor 2500 punten)}
\end{figure}

\begin{figure}
\includegraphics[width=\textwidth]{dim-Kavg.png}
\caption{De plot tussen het aantal dimensies en Kavg (voor 2500 punten)}
\end{figure}

\begin{figure}
\includegraphics[width=\textwidth]{dim-Kmax.png}
\caption{De plot tussen het aantal dimensies en Kmax (voor 2500 punten)}
\end{figure}
\end{document}
